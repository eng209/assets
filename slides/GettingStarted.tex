% !TEX TS-program = xelatex
% !TEX encoding = UTF-8 Unicode
% -*- coding: UTF-8; -*-
% vim: set fenc=utf-8

\documentclass[10pt]{article}
%\usepackage{fontspec,xltxtra,xunicode}
%\defaultfontfeatures{Mapping=tex-text}


\usepackage{microtype}
\usepackage{newlfont}
\usepackage{graphicx}
\usepackage{a4wide}
\usepackage{geometry}
\geometry{a4paper, top=1.5cm, left=2cm, right=3.0cm, bottom=1.5cm}

\usepackage{amssymb}
\usepackage{amsmath}
\usepackage{enumerate}
\usepackage{mathrsfs}
\usepackage{thumbpdf}
\usepackage{charter}
\usepackage[xetex,raiselinks=false,colorlinks=true,urlcolor=blue,linkcolor=black]{hyperref}
\renewcommand\UrlFont{\sffamily}
%\usepackage[none]{hyphenat}
\usepackage{ulem}

%\setmainfont[Mapping=tex-text]{Palatino}
%\setsansfont[Mapping=tex-text]{Helvetica Neue}
%\newfontfamily{\monofontstraightquotes}[Mapping=tex-ansi, Scale=0.8]{Bitstream Vera Sans Mono}
%\setmonofont[Mapping=tex-text, Scale=0.9]{Bitstream Vera Sans Mono}
\newcommand{\squot}{{\monofontstraightquotes "}}
\newcommand{\sapos}{{\monofontstraightquotes '}}

\usepackage{listings}
\usepackage{color}
\usepackage{framed}
\usepackage{titlesec}
\usepackage{textcomp}

\usepackage{etoolbox}
\usepackage{pythonhighlight}

\begin{document}
\setlength{\parskip}{5pt plus 1pt minus 1pt}
\setlength{\parindent}{0pt}
\setlength{\marginparsep}{0.5cm}
\def\thesection{Exercice \arabic{section}.}

\DeclareRobustCommand{\cmdkey}{\raisebox{-.06em}{\includegraphics[height=.75em]{command.pdf}}}

\newcommand\•{$\cdot$}
\newcommand\code[1]{\texttt{\textbf{\small #1}}}
\newcommand\ui[1]{\textsf{#1}}
\newcommand{\marginnote}[1]{\marginpar{\flushleft{{\scriptsize ~\\} \textsf{#1}}}}

%%%%%%%%%%%%%%%%%%%%%%%%%%%%%%%%%%%%%%%%%%%%%%%%

{
	\sffamily
	{\Large \bfseries ENG-209 --- Getting Started}
	{\hfill 9 septembre 2024}

}


\vspace{0.1cm}
\hrule
\vspace{0.3cm}


%%%%%%%%%%%%%%%%%%%%%%%%%%%%%%%%%%%%%%%%%%%%%%%%

%\section{Remarques préliminaires et configuration initiale}

\newcommand{\bigsubsection}[1]{
	\vspace{0.7cm}
	\textbf{\textit{\large #1}}
}
Ici, vous allez configurer votre machine de travail pour faire les séries d'exercices du semestre.

\subsubsection*{Connexion}

Connectez-vous sur l'infrastructure des postes de travail virtuels:
\begin{itemize}
\itemsep0em
\item avec les machines en salle INF3: identifiez-vous directement avec votre identifiant GASPAR;
\item sur votre propre machine: en téléchargeant d'abord VMWare Horizon Client depuis \url{https://vdi.epfl.ch}, puis en procédant comme pour la salle INF3;
\end{itemize}

… puis choisissez la machine virtuelle \code{IC-CO-IN-INJ-2024-Fall} (et pas une autre)

Notez bien qu'à chaque logout, vos données sur l’ordinateur sont \textbf{effacées}. Seulement le contenu de votre dossier \code{myfiles}, visible sur votre bureau après l'ouverture de la machine virtuelle, est sauvegardé et réapparaît au prochain login.

Faites donc attention à toujours travailler dans votre dossier réseau!

Vous pouvez aussi accéder à \code{myfiles} en le montant comme dossier réseau sur votre propre machine. Si vous n'êtes pas sur le réseau de l'EPFL, vous devrez vous connecter au VPN d'abord. Plus d'info: \url{http://mynas.epfl.ch}; \url{http://studinfo.epfl.ch/core/index.asp?article=18}; \url{https://vpn.epfl.ch}.

\subsubsection*{Configuration}

Voici la procédure pour configurer le tout sur un poste de travail virtuel. \textbf{Ceci, sauf indication contraire, est à faire une seule fois lors de la première séance d'exercices:}
\begin{enumerate}
\item Loggez-vous sur une machine virtuelle (via une des machines de la salle d'exercice ou via votre propre machine par l'intermédiaire de VMware Horizon Client comme indiqué sur \url{https://vdi.epfl.ch}).
\item Lancez \ui{Firefox} depuis la barre latérale, puis allez sur la page Moodle du cours et téléchargez (dans votre dossier \ui{Downloads}, par défaut) le fichier de configuration \code{setup.sh}.
\item Ouvrez l'application \ui{Terminal} (par exemple via le lanceur en bas de la barre latérale), puis tapez exactement ceci, ligne par ligne et en faisant bien attention aux espaces:\\
\code{cd Downloads}\\
\code{chmod +x setup.sh}\\
\code{./setup.sh}
\item Observez la machine travailler pour vous. Cela prend un certain temps. À la fin de l'opération, s'il n'y a pas eu d'erreur, fermez la fenêtre du terminal puis lancez \ui{Visual Studio Code} (aussi depuis le lanceur en bas à gauche). Choisissez ensuite \ui{File} $\rightarrow$ \ui{Open Folder}, puis naviguez vers dossier workspace, qui ressemble à ceci:\\
\code{~/Desktop/myfiles/ENG209\_XXXX/}.\\
Attention à bien ouvrir ce dossier et pas myfiles directement ou un autre.
\item Chaque semaine, récupérez les séries et les éventuels corrigés depuis Moodle et glissez-les dans le dossier de votre workspace.
\end{enumerate}

Rappel: cette procédure est à effectuer uniquement la première fois que vous accédez à votre machine virtuelle. Les fois suivantes, vous ouvrez directement VS Code, puis faites \ui{Open Workspace…} si nécessaire pour retrouver l'ensemble de vos fichiers de travail.

Si vous souhaitez ne pas travailler sur cette infrastructure, c'est possible de tout faire en local sur votre propre machine. Mais c'est à vous de la configurer; nous vous aidons dans la mesure du possible mais ce n'est pas une configuration officiellement «prise en charge». Sachez aussi que l'examen de mi-semestre se déroulera obligatoirement sur les machine virtuelles avec les postes de l'EPFL afin que les conditions de passation soient identiques pour tout le monde.


\end{document}
